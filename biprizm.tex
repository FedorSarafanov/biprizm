% Тип документа
\documentclass[a4paper,12pt]{extarticle}

% Шрифты, кодировки, символьные таблицы, переносы
\usepackage{cmap}
\usepackage[T2A]{fontenc}
\usepackage[utf8x]{inputenc}
\usepackage[russian]{babel}

% Это пакет -- хитрый пакет, он нужен но не нужен
\usepackage[mode=buildnew]{standalone}

\usepackage
	{
		% Дополнения Американского математического общества (AMS)
		amssymb,
		amsfonts,
		amsmath,
		amsthm,
		% misccorr,
		% 
		% Графики и рисунки
		wrapfig,
		graphicx,
		subcaption,
		float,
		tikz,
		tikz-3dplot,
		caption,
		csvsimple,
		color,
		booktabs,
		pgfplots,
		pgfplotstable,
		geometry,
		% 
		% Таблицы, списки
		makecell,
		multirow,
		indentfirst,
		%
		% Интегралы и прочие обозначения
		ulem,
		esint,
		esdiff,
		% 
		% Колонтитулы
		fancyhdr,
	}  


% Обводка текста в TikZ
\usepackage[outline]{contour}

% Увеличенный межстрочный интервал, французские пробелы
\linespread{1.3} 
\frenchspacing 

 
\usetikzlibrary
	{
		decorations.pathreplacing,
		decorations.pathmorphing,
		patterns,
		calc,
		scopes,
		arrows,
		fadings,
		through,
		shapes.misc,
		arrows.meta,
		3d,
		quotes,
		angles,
		babel
	}


\tikzset{
	force/.style=	{
		>=latex,
		draw=blue,
		fill=blue,
				 	}, 
	%				 	
	axis/.style=	{
		densely dashed,
		blue,
		line width=1pt,
		font=\small,
					},
	%
	th/.style=	{
		line width=1pt},
	%
	acceleration/.style={
		>=open triangle 60,
		draw=magenta,
		fill=magenta,
					},
	%
	inforce/.style=	{
		force,
		double equal sign distance=2pt,
					},
	%
	interface/.style={
		pattern = north east lines, 
		draw    = none, 
		pattern color=gray!60,
					},
	cross/.style=	{
		cross out, 
		draw=black, 
		minimum size=2*(#1-\pgflinewidth), 
		inner sep=0pt, outer sep=0pt,
					},
	%
	cargo/.style=	{
		rectangle, 
		fill=black!70, 
		inner sep=2.5mm,
					},
	%
	caption/.style= {
		midway,
		fill=white!20, 
		opacity=0.9
					},
	%
	}

\newenvironment{tikzpict}
    {
	    \begin{figure}[htbp]
		\centering
		\begin{tikzpicture}
    }
    { 
		\end{tikzpicture}
		% \caption{caption}
		% \label{fig:label}
		\end{figure}
    }


\newcommand{\vbLabel}[3]{\draw ($(#1,#2)+(0,5pt)$) -- ($(#1,#2)-(0,5pt)$) node[below]{#3}}
\newcommand{\vaLabel}[3]{\draw ($(#1,#2)+(0,5pt)$) node[above]{#3} -- ($(#1,#2)-(0,5pt)$) }

\newcommand{\hrLabel}[3]{\draw ($(#1,#2)+(5pt,0)$) -- ($(#1,#2)-(5pt,0)$) node[right, xshift=1em]{#3}}
\newcommand{\hlLabel}[3]{\draw ($(#1,#2)+(5pt,0)$) node[left, xshift=-1em]{#3} -- ($(#1,#2)-(5pt,0)$) }



\newcommand\zi{^{\,*}_i}
\newcommand\sumn{\sum_{i=1}^{N}}

\tikzset{
	coordsys/.style={scale=1.8,x={(1.1cm,-0cm)},y={(0.5cm,1cm)}, z={(0cm,0.8cm)}},
	coordsys/.style={scale=1.5,x={(0cm,0cm)},y={(1cm,0cm)}, z={(0cm,1cm)}}, 
	coordsys/.style={scale=1.5,x={(1cm,0cm)},y={(0cm,1cm)}, z={(0cm,0cm)}}, 
}

\usepgfplotslibrary{units}


% Draw line annotation
% Input:
%   #1 Line offset (optional)
%   #2 Line angle
%   #3 Line length
%   #5 Line label
% Example:
%   \lineann[1]{30}{2}{$L_1$}

\newcommand{\lineann}[4][0.5]{%
    \begin{scope}[rotate=#2, blue,inner sep=2pt, ]
        \draw[dashed, blue!40] (0,0) -- +(0,#1)
            node [coordinate, near end] (a) {};
        \draw[dashed, blue!40] (#3,0) -- +(0,#1)
            node [coordinate, near end] (b) {};
        \draw[|<->|] (a) -- node[fill=white, scale=0.8] {#4} (b);
    \end{scope}
}

\newcommand{\lineannn}[4][0.5]{%
    \begin{scope}[rotate=#2, blue,inner sep=2pt, ]
        \draw[dashed, blue!40] (0,0) -- +(0,#1)
            node [coordinate, near end] (a) {};
        \draw[dashed, blue!40] (#3,0) -- +(0,#1)
            node [coordinate, near end] (b) {};
        % \draw[color=white, color=blue] (a) -- node[fill=white, scale=0.8] {#4} (b);
        \draw[->|] (a)++(-0.3,0) -- (a);
        \draw[->|] (b)++(0.3,0) coordinate (xx) -- (b);
        \draw (xx) node[fill=white, scale=0.8, right] {#4};
    \end{scope}
}

% Круговая стрелка относительно центра (дуга из центра)
\tikzset{
  pics/carc/.style args={#1:#2:#3}{
    code={
      \draw[pic actions] (#1:#3) arc(#1:#2:#3);
    }
  },
  dash/.style={
  	dash pattern=on 5mm off 5mm
  }
}

% Среднее <#1>
\newcommand{\mean}[1]{\langle#1\rangle}

\pgfplotsset{
    % most recent feature set of pgfplots
    compat=newest,
}

% const прямым шрифтом
\newcommand\ct[1]{\text{\rmfamily\upshape #1}}
\newcommand*{\const}{\ct{const}}


\usepackage[europeanresistors,americaninductors]{circuitikz}

% Style to select only points from #1 to #2 (inclusive)
\pgfplotsset{select/.style 2 args={
    x filter/.code={
        \ifnum\coordindex<#1\def\pgfmathresult{}\fi
        \ifnum\coordindex>#2\def\pgfmathresult{}\fi
    }
}}


\usepackage{array}
\usepackage{pstool}


%%%%%%%%%%%%%%%%%%%%%%%%%%%%%%%%%%%%%%%%%%%%%%%%%
\makeatletter
\newif\if@gather@prefix 
\preto\place@tag@gather{% 
  \if@gather@prefix\iftagsleft@ 
    \kern-\gdisplaywidth@ 
    \rlap{\gather@prefix}% 
    \kern\gdisplaywidth@ 
  \fi\fi 
} 
\appto\place@tag@gather{% 
  \if@gather@prefix\iftagsleft@\else 
    \kern-\displaywidth 
    \rlap{\gather@prefix}% 
    \kern\displaywidth 
  \fi\fi 
  \global\@gather@prefixfalse 
} 
\preto\place@tag{% 
  \if@gather@prefix\iftagsleft@ 
    \kern-\gdisplaywidth@ 
    \rlap{\gather@prefix}% 
    \kern\displaywidth@ 
  \fi\fi 
} 
\appto\place@tag{% 
  \if@gather@prefix\iftagsleft@\else 
    \kern-\displaywidth 
    \rlap{\gather@prefix}% 
    \kern\displaywidth 
  \fi\fi 
  \global\@gather@prefixfalse 
} 
\newcommand*{\beforetext}[1]{% 
  \ifmeasuring@\else
  \gdef\gather@prefix{#1}% 
  \global\@gather@prefixtrue 
  \fi
} 
\makeatother
%%%%%%%%%%%%%%%%%%%%%%%%%%%%%%%%%%%%%%%%%%%%%%%%%

\geometry		
	{
		left			=	2cm,
		right 			=	2cm,
		top 			=	3cm,
		bottom 			=	3cm,
		bindingoffset	=	0cm
	}

%%%%%%%%%%%%%%%%%%%%%%%%%%%%%%%%%%%%%%%%%%%%%%%%%%%%%%%%%%%%%%%%%%%%%%%%%%%%%%%



	%применим колонтитул к стилю страницы
\pagestyle{fancy} 
	%очистим "шапку" страницы
\fancyhead{} 
	%слева сверху на четных и справа на нечетных
\fancyhead[R]{\labauthors} 
	%справа сверху на четных и слева на нечетных
\fancyhead[L]{Отчёт по лабораторной работе №\labnumber} 
	%очистим "подвал" страницы
\fancyfoot{} 
	% номер страницы в нижнем колинтуле в центре
\fancyfoot[C]{\thepage} 

%%%%%%%%%%%%%%%%%%%%%%%%%%%%%%%%%%%%%%%%%%%%%%%%%%%%%%%%%%%%%%%%%%%%%%%%%%%%%%%

\renewcommand{\contentsname}{Оглавление}

\usepackage{tocloft}
% \renewcommand{\cftpartleader}{\cftdotfill{\cftdotsep}} % for parts
% \renewcommand{\cftsectiondotsep}{\cftdotsep}% Chapters should use dots in ToC
\renewcommand{\cftsecleader}{\cftdotfill{\cftdotsep}}
%\renewcommand{\cftsecleader}{\cftdotfill{\cftdotsep}} % for sections, if you really want! (It is default in report and book class (So you may not need it).
% ---------
% \newcommand{\cftchapaftersnum}{.}%
% \usepackage{titlesec}
% \titlelabel{\thetitle.\quad}
\usepackage{secdot}
\sectiondot{subsection}
\usepackage{setspace}

\begin{document}

\def\labauthors{Понур К.А., Сарафанов Ф.Г., Сидоров Д.А.}
\def\labgroup{420}
\def\labnumber{999}
\def\labtheme{Изучение интерференции в схеме с бипризмой Френеля}
\renewcommand{\vec}{\mathbf}
\renewcommand{\Re}{\operatorname{Re}}
\renewcommand{\Im}{\operatorname{Im}}
\renewcommand{\phi}{\phi}
\renewcommand{\kappa}{\varkappa}
\renewcommand{\hat}{\widehat}
\renewcommand{\epsilon}{\varepsilon}
\renewcommand{\phi}{\varphi}
%%%%%%%%%%%%%%%%%%%%%%%%%%%%%%%%%%%%%%%%%%%%%%%%%%%%%%%%%%%%%%%%%%%%%%%%%%%%%%%
\begin{titlepage}

\begin{center}

{\small\textsc{Нижегородский государственный университет имени Н.\,И. Лобачевского}}
\vskip 1pt \hrule \vskip 3pt
{\small\textsc{Радиофизический факультет}}

\vfill

{\Large Отчет по лабораторной работе №\labnumber\vskip 12pt\bfseries \labtheme}
	
\end{center}

\vfill
	
\begin{flushright}
	{Выполнили студенты \labgroup\ группы\\ \labauthors}%\vskip 12pt Принял:\\ Менсов С.\,Н.}
\end{flushright}
	
\vfill
	
\begin{center}
	Нижний Новгород, \the\year
\end{center}

\end{titlepage}


%%%%%%%%%%%%%%%%%%%%%%%%%%%%%%%%%%%%%%%%%%%%%%%%%%%%%%%%%%%%%%%%%%%%%%%%%%%%%%%
\begin{spacing}{1}
\tableofcontents
\end{spacing}
% \setstretch{1.2}
\newpage
%%%%%%%%%%%%%%%%%%%%%%%%%%%%%%%%%%%%%%%%%%%%%%%%%%%%%%%%%%%%%%%%%%%%%%%%%%%%%%%


 \section{}
\subsection{Введение}

Цель работы -- целью данной работы является получение интерфереционной картины, проверка некоторых теоретических формул
и определение средней длины волны света, пропускаемого красным и зеленым светофильтрами.
В данной работе для получения когерентных источников света применяется способ, предложенный Френелем и связанный с использованием бипризмы.
\subsection{Теоретическая часть}
В произвольной точке экрана результирующая интенсивность $I(x)$ есть усредненное за время регистрации $\tau$ значение квадрата напряженности суммарного электрического поля:
\begin{gather}
\label{eq:1}
	\vec{E}(r,t)=\vec{E_1}(r_1,t)+\vec{E_2}(r_2,t)=\\=
	-\vec{A_1}(r_1)\cos(\omega t-kr_1+\phi_1)+\vec{A_2}(r_2)
		\cos(\omega t -kr_2+\phi_2), \text{то есть} \nonumber
\end{gather}
\begin{equation}
	I(x)=A_1^2+A_2^2+2(\vec{A_1},\vec{A_2})\cos[k(r_2-r_1-(\phi_2-\phi_1)]
\end{equation}
Бипризма представляет собой две соединенные своими основаниями призмы с одинаковыми и очень малыми (порядка долей градуса) преломляющими углами. 

Каждая из половинок бипризмы отклоняет падающие на неё лучи к своему основанию и поворачивает тем самым фронт волны.  Продолжения лучей, отклоненных первой половиной бипризмы, пересекаются в точке $S_1$, которую можно рассматривать как мнимый источник света. Продолжения всех лучей, отконенных второй половиной бипризмы, пересекаются в точке $S_2$, которую можно рассматривать как другой мнимый источник света. Так как лучи, отклоненные обеими половинками бипризмы, падают на неё от одногои того же источника света, то мнимые источники света $S_1$ и $S_2$ будут когерентны. 

Та область, в которой распространяет- ся волне, отклоненная одной только первой половиной бипризмы, на рис. 3 заштрихована линиями, параллельными $OA$. Та область, в которой распространяется волна, отклоненная одной только второй половиной бипризмы, заштрихована линиями, параллельными $OB$. В области $OMN$ , покрытой на рис. 3 двойной
штриховкой, происходит наложение двух когерентных волн от двух мнимых источников $S_1$, и $S_2$. В этой области пространства имеют место явления интерференции и на участке $MN$ экрана
наблюдения мы увидим ряд светлых и темных (при освещении белым светом - окрашенных) интерференционных полос.

При построении хода лучей, отклоняемых бипризмой (см рис 3) в случае малого преломляющего угла ы. бипризмы и малых углов падения лучей на призму можно воспользоваться следующей приближенной формулой для угла отклонения $\epsilon$
Согласно этому выражению угол отклонении призмой лучей в рассматриваемом приближении не зависит от угля падения и целиком определяется материалом и геометрией призмы. Так, например, если показатель преломления стекла, из которого сделана бипризма, $n=1.5$, то угол отклонения $\epsilon$ просто равен половине преломляющего угла $\alpha$ призмы:
\begin{equation}
 	\epsilon=\frac{\alpha}{2}
 \end{equation} 
Воспользовавшись формулой $s$ или $s$ и выполнив построение хода лучей, можно убедиться в том, что, если $SO\bot AB$ (см.рис. 3), то мнимые изображения	и	действительного источника света $S$ лежат в одной плоскости с действительным источником, причем эта плоскость параллельна передней грани бипризмы. Это обстоятельство в дальнейшем облегчит нам нахождение расстояния $\delta$ между мнимыми источниками $S_1$ и	$S_2$.
Ограничения поля интерференции $MN$ за бипризмой зависят от величины предельного угла расходимости $\phi_0$ светового пучка, падающего на бипризму от щели $S$. Особый интерес представляют два частных случая:

1. При $\phi_0 = 2\epsilon$	линейная ширина поля интерференции,
начиная с расстояния $h$ за бипризмой, остается неизменной и равна расстоянию $\delta$ между мнимыми источниками $S_1$ и $S_2$.

2. При $h\rightarrow\infty$,что можно осуществить,
осветив бипризму параллельным пучком лучей, полученным с помощью вспомогательной линзы (см. рис. 4), сечение поля интерференции имеет форму ромба. Максимальная ширина поля интерференции $MN$ в этом случае равна половине ширины параллельного пучка падающего на бипризму. Такая схема интерференции соответствует cлучаю наложения двух параллельных когерентных световых пучков
пересекающих друг друга под постоянным углом.

Для расчета наблюдаемой на экране интерференционной картины воспользуемся тем, что бипризма Френеля так изменяет ход лучей от действительного источника, что дает нам право рассматривать световое возмущение в области $MN$ (рис. 3) как результат синфазного излучения двух мнимых источников $S_1$ и $S_2$. При этом рассматривая выражение (\ref{eq:1})для соответствующих проекций $vec{E_1}(r,t)$ $vec{E_2}(r,t)$, пренебрежем зависимостью амплитуд $A_1$ и $A_2$ от расстояния $r$, то есть будем считать $A_1=A_2=A_0$	и положим $\phi_1=\phi_2=0$.

Согласно обозначениям, приведенным на рис. , найдем как ширина $d$ полос интерференции зависит от параметров нашей измерительной установки, то есть от длины установки $L$,расстояния между мнимыми источниками и длины волны света  
$\lambda$, испускаемого действиткльным источником $S$. В точку $P$ на экране $MN$ колебания источников $S_1$ и $S_2$ придут с разностью хода:
\begin{equation}
	\Delta=S_2B=r_2-r_1
\end{equation}
и, следовательно, с разностью фаз
\begin{equation}
	\phi(x)=\frac{2\pi}{\lambda}(r_2-r_1)
\end{equation}
На основании вышеизложенного и в соответствии с выражением (\ref{eq:1}) интенсивность результирующего колебания в точке наблюдения $P$ с координатой $x$ определяется формулой
\begin{equation}
	I(x)=2A^2[1+\cos{\phi(x)}]=A^2\cos^2{\frac{\phi}{2}}
\end{equation}
Максимумы освещенности будут получаться в тех местах экрана, для которых разность фаз
\begin{equation}
	\phi(x)=\frac{2\pi}{\lambda}\Delta=2\pi m, \text{где} m=0;\pm 1,\pm 2,\cdots
\end{equation}
То есть для которых разность хода
\begin{equation}
	\Delta=r_2-r_1=m\lambda
\end{equation}

Для нахождения координат максимумов интенсивности вычислим разность хода $\Delta=r_2-r_1$. Следуя обозначениям на рис.(), получаем, что 
\begin{gather}
	\label{eq:8}
	r_2-r_1=\frac{4ax}{r_1+r_2} \\ \nonumber
	r_1^2=L^2+(x-a)^2 \\ \nonumber
	r_2^2=L^2+(x+a)^2 \nonumber
\end{gather}


Предполагая величины $\frac{x+a}{L}$ и $\frac{x-a}{L}$ малыми, разложим $r_1$ и $r_2$ в ряд и ограничимся двумя членами в разложении. В результате получим
\begin{equation}
	\label{eq:9}
	r_1+r_2\simeq 2L +\frac{x^2+a^2}{L} 
\end{equation}
Подставляя (\ref{eq:9}) в (\ref{eq:8}) найдём, что 
\begin{equation}
	r_2-r_1\simeq \frac{2ax}{L}\left(1-\frac{x^1+a^2}{2l^2} \right) \label{eq:10}
\end{equation}
При условии 
\begin{equation}
	\frac{\delta x(x^2+a^2)}{2L^3}\ll\frac{\lambda}{2}
\end{equation}
которое позволяет в выражении для разности хода (\ref{eq:10}) отбросить слагаемое, дающее малый по сравнению с $\pi$ вклад в разность фаз интерфеиррующих волн, точное выражение (\ref{eq:8}) может быть заменено на приближенное
\begin{equation}
 	r_1-r_2\simeq\frac{\delta x}{L} \label{eq:12}
 \end{equation} 
 Отметим, что выражение (\ref{eq:12}) сразу следует при условии малости угла 
 $\theta (\sin{\theta}\simeq\theta)$ из приближения приближения парралельных лучей. 
 \begin{equation}
 	\Delta=S_2C=\delta\sin{\theta}\simeq\frac{\delta x}{L}
 \end{equation}
 Следовательно, ширина полос интерференции, равная расстоянию между двумя соседними максимумами освещенности в первои приближении равна:
 \begin{equation}
 	x_{m+1}-x_m=d=\frac{L\lambda}{\delta} \label{eq:14}
 \end{equation}
 Формулу (\ref{eq:14}), переписанную в другом виде
 \begin{equation}
 	\delta d=L\lambda
 \end{equation}
 удобно использовать для проверки теории интерференционных явлений. Если оставлять неизменным расстояние $L$ между щелью $S$ и экраном наблюдения и работать с одной и той же длиной волны $\lambda$ (пользоваться одним и тем же светофильтром), то произведение $\delta d$ должно оставаться (согласно теории) постоянным. Таким образом, для проверки теории нужно, меняя расстояние между мнимыми источниками, независимыми способами измерять расстояния $\delta$ и $d$.Если их произведение будет оставаться постоянным (конечно, при $L= \const$ и $\lambda=\const$ ), то это будет служить доказательством правильности изложенной теории. Расстояние $\delta$ между мнимыми источниками в данной работе можно изменять, изменяя величину $h$ (см. рис. ). То есть помещая бипризму на различным расстояниях от щели.
%%%%%%%%%%%%%%%%%%%%%%%%%%%%%%%%%%%%%%%%%%%%%%%%%%%%%%%%%%%%%%%%%%%%%%%%%%%%%%%
\subsection{Собственные колебания в электрическром контуре}

\section{Заключение}

\end{document} 